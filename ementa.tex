% -*- andrei: LuaLaTeX -*-
\documentclass[12pt]{article}
\input{/home/andrei/a/other/emacs/tex-outline-hat.tex}

\usepackage{fontspec}


% NOTE: C-c C-c bound to LuaLaTeX

\begin{document} 
\section{Andrey Yuryevich Mikhaylov. Proposta de disciplina para o ano de 2013}

Proponho uma nova disciplina: 
\begin{center}
{\bf Geometria diferencial e mecânica clássica}
\end{center}

\subsection{Justificativa}
Neste ano dei aulas da Mecânica Clássica. Só assistiram 3 alunos + 1 
voluntário. Acontece que a maioria dos alunos já tiveram algum curso da 
Mecânica Clássica, principalmente usando o livro do Prof. Goldstein. Até os 
meus 3 alunos já assistiram o curso de Goldstein. Então resolvi usar o livro 
do Arnold, e tratei principalmente as applicações da geometria diferencial na 
mecânica clássica.

\subsection{Proposta}
No futuro, proponho {\bf mudar o currículo do IFT}:

\begin{enumerate}
   \item Cancelar o curso da Mecânica Clássica, porque a grande maioria dos 
      alunos já sabem tudo isso
   \item Em vez de Mecânica Clássica, introduzir a disciplina nova:
      \begin{itemize}
      \item {\large \bf Geometria diferencial e mecânica clássica}
      \end{itemize}
      Esta disciplina servirá introdução à geometria diferencial, 
      principalmente os apectos dela mais usados na Física Teórica. Também, 
      para aqueles alunos que precisarem, vou explicar todos os elementos 
      necessários da mecânica clássica.
   \item Aquela nova disciplina pode ser pré-requisito 
      para ``Relatividade Geral'', e talvez para
      ``Introdução a Supersimetria''.
\end{enumerate}
Esta disciplina vai desenvolver as duas disciplinas que já ensinei no IFT,
``Differential geometry for physics'' e ``Mecânica Clássica''.

\eject

\section{O plano da disciplina}
\begin{enumerate}
\item Variedades diferenciáveis
   \begin{enumerate}
   \item Definição e exemplos
   \item Espaço tangente
   \item Formas diferenciais
      \begin{itemize}
      \item Formas diferenciais e difeomorfismos
      \item Integração
      \item Derivada exterior e formula de Stokes
      \item Derivada de Lie e a sua relação com a derivada exterior
      \item Cohomologia de de Rham
      \end{itemize}
   \end{enumerate}
\item Fluxo 1-parametrico
   \begin{enumerate}
   \item Definição e relação com a teoria das equações diferenciais ordinárias
   \item Grupo de difeomorfismos
   \item Correspondência entre campos vetoriais, fluxos, e operadores 
      diferenciais lineares
   \end{enumerate}
\item Grupos e álgebras de Lie
   \begin{enumerate}
   \item Definição de Grupo de Lie e o papel deles na Física Teórica
   \item Álgebras de Lie
      \begin{itemize}
      \item Relação entre álgebras de Lie e grupos de Lie
      \item Álgebras de Lie abstratos
      \item Álgebras de campos vetoriais; interpretação algébrica da derivada
         exterior
      \end{itemize}
   \item Espaços homogénicos
   \end{enumerate}
\item Princípio de Mínima Ação
   \begin{enumerate}
   \item Ação, princípio variacional, equações de movimento
   \item Teorema de Noether
   \end{enumerate}
\item Mecânica Hamiltoniana
   \begin{enumerate}
   \item Variedades simpléticas
      \begin{itemize}
      \item Definição da estrutura simplética
      \item Definição da estrutura de Poisson e a sua relação com a estrutura
         simplética
      \item Identidade de Jacobi e a sua interpretação geométrica
      \end{itemize}
   \item Derivação do formalismo Hamiltoniano a partir do princípio 
      de Mínima Ação
      \begin{itemize}
      \item Espaço de fase é o espaço de trajetórias clássicas
      \item A forma simplética do ponto de vista do princípio variacional
      \item Campos vetoriais Hamiltonianos
      \end{itemize}
   \item Geometria do espaço de fase
      \begin{itemize}
      \item Teorema de Darboux
      \item Espaço de fase aumentado e invariantos integrais
      \item Transformações canónicas; transformações canónicas que incluem o 
         tempo
      \item Estruturas de Poisson degeneradas, folhas simpléticas, redução 
         Hamiltoniana
      \end{itemize}
   \end{enumerate}
\item Metodo de Hamilton e Jacobi
   \begin{enumerate}
   \item Equações diferenciais quase-lineares
   \item Equações diferenciais non-lineares
   \item Estruturas de contato
   \item Superfícies envelopantes e características
   \item Interpretação do ponto de visto das transformações canónicas
   \item Princípio de Maupertuis
   \end{enumerate}
\item Sistemas integraveis e suas perturbações
   \begin{enumerate}
   \item Teorema de Liouville
   \item Relação com a teoria de Hamilton e Jacobi
   \item Prova da teorema de Liouville
   \item Variaveis de ação: prova da existência
   \item Variaveis de ação: construção
   \item Invariantos adiabáticos
   \item Teoría de perturbações; discução da teoria KAM
   \end{enumerate}
\end{enumerate}

\section{Demais informações}
Number of credits: 12 (4 mêses)

\section{Bibliografia}
Nós vamos usar o livro do V.I.Arnold 
``Mathematical methods of classical mechanics'', mas com muito preocupação. 
Este livro é escrito para matemáticos. Para que usá-lo no ensino da Física 
Teórica, tive que modificá-lo consideravelmente. Principalmente, o curso dele
tem os problemas seguintes:
\begin{enumerate}
\item O objetivo (quase principal) dele é desenvolver a Mecânica Clássica
   para usá-la como a ferramenta para provar teorémas matemáticas. Nós não
   temos aquele objetivo, temos outros objetivos
\item As vezes, ele introduze os objetos do jeito ``axiomático'', sem nenhuma
   lógica. No ensino da Física Teórica, nós temos que seguir a lógica interna
   da nossa disciplina, sempre explicando de onde as coisas vêm
\item Acho que as vezes, as provas dele são demais complicadas; isso pode ser
   necessário no ensino matemático, para conseguir o rigor absoluto; nós não
   temos tal objetivo
\end{enumerate}
Acho que consegui, pelo menos parcialmente, corrigir estas faltas. 

Nós vamos também usar algums capítulos do outro livro dele que se chama 
``Ordinary differential equations''. 

Também vamos usar algums capítulos do livro: S.P.Novikov 
``Modern geometry: Methods and Applications''.

\end{document}